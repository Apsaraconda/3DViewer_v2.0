\documentclass{report}
% подключаем русский шрифт
\usepackage[utf8]{inputenc}
\usepackage[russian]{babel}
\righthyphenmin=2
\voffset = 0pt
\topmargin = 0pt
\headsep = 0pt
\headheight = 0pt
\textheight = 700pt
\oddsidemargin = 31pt
\marginparsep = 0pt
\marginparwidth = 0pt
\textwidth = 460pt

% начинаем документ
\begin{document}
\section* {\bfseries 3D\_Viewer\_v2.0}
\section* {\bfseries Краткое описание возможностей программы.}
\begin{itemize}
\item[1)] Чтобы открыть файл нажмите File-> Open file...

\item[2)] Дополнительные настройки цвета фона, отображения вершин,
  выбора типа проекции находятся в меню Edit->Preferences

\item[3)] Информация о загруженной модели находится в нижней
  части приложения

\item[4)] Модель можно вращать, перемещать, масштабировать
  эти настройки находятся в правой части окна программы

\item[5)] Для сохранения картинки или создания gif- анимации
  предусмотрен выбор типа сохраняемого файла и кнопка
  Save. При сохранении gif-анимации во время создания
  анимации, модель следует вращать или перемещать
\end{itemize}
\section* {\bfseries Установка.}

Чтобы начать использовать 3D\_Viewer, его необходимо установить с помощью команды make install. Эта команда создает папку 3D\_Viewer\_v2.0 в текущей директории и устанавливает программму в неё.

\section* {\bfseries Удаление.}

Удалить приложение можно с помощью команды make uninstall.

\section* {\bfseries ZIP.}

Архивировать проект можно с помощью команды make dist.

\section* {\bfseries Основные особенности:}
\begin{itemize}
\item Программа реализована с использованием паттерна MVC;

\item Загрузка и просмотр каркасной модели из файла формата obj (поддержка только списка вершин и поверхностей);

\item Перемещение модели на заданное расстояние относительно осей X, Y, Z;

\item Поворот модели на заданный угол относительно своих осей X, Y, Z;

\item Масштабирование модели на заданное значение;

\item Программа позволяет настраивать тип проекции (параллельная и центральная);

\item Программа позволяет настраивать тип (сплошная, пунктирная), цвет и толщину ребер, способ отображения (отсутствует, круг, квадрат), цвет и размер вершин;

\item Программа позволяет выбирать цвет фона;

\item Настройки сохраняются между перезапусками программы в файле settings.ini;

\item Программа позволяет сохранять полученные ("отрендеренные") изображения в файл в форматах bmp и jpeg;

\item Программа позволяет по кнопке Save записывать небольшие "скринкасты" - текущие пользовательские аффинные преобразования загруженного объекта в gif-анимацию (640x480, 10fps, 5s);

% Скобочные арифметические выражения в инфиксной нотации должны поддерживать следующие арифметические операции и математические функции:
\end{itemize}
\end{document}